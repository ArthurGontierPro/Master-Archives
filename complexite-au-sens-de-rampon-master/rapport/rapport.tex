% Document : compte rendu des DM
% Auteur : Xavier Gandibleux
% Année académique : 2018-2019

\documentclass[a4paper,10pt]{article}
\usepackage{tikz}


% passe en mode large sur la page A4
\usepackage{a4wide} 

% document francisé
\usepackage[polutonikogreek,english,francais]{babel} 

% permet la frappe de caracteres accentues (sur macOS)
%\usepackage[utf8x]{inputenc} 

\usepackage{graphicx,float,subcaption} % figure et placement de figure
\usepackage[top=10mm, bottom=10mm, left=20mm, right=20mm]{geometry}

\usepackage[linesnumbered,frenchkw,ruled,french]{algorithm2e}


\usepackage[T1]{fontenc}
\usepackage{dsfont}
%\usepackage{graphicx}
%\usepackage{grffile}
%\usepackage{longtable}
%\usepackage{wrapfig}
%\usepackage{rotating}
%\usepackage[normalem]{ulem}
%\usepackage{amsmath}
%\usepackage{amsthm}
%\usepackage{textcomp}
%\usepackage{amssymb}
%\usepackage{capt-of}
\usepackage{hyperref}% \documentclass[a4paper,12pt]{article}
%\usepackage[margin=2cm]{geometry}
\usepackage[utf8]{inputenc}

\usepackage{color}
\usepackage{xcolor}

\usepackage{amsmath,amsfonts,amssymb,stmaryrd,amsthm}
%\usepackage{fullpage}
%\usepackage{multirow}
%\usepackage[rounded]{syntax}
%\usepackage[section]{placeins}
\newtheorem{example}{Exemple} %%% modifier ici si on veut en fr/en

\usepackage{todonotes}
\usepackage{listings} 


\usepackage{listings}
\usepackage{inconsolata} % very nice fixed-width font included with texlive-full
%\usepackage[usenames,dvipsnames]{color} % more flexible names for syntax highlighting colors

\definecolor{chameleond}{HTML}{4E9A06}
\definecolor{skyblued}{HTML}{204A87}
\definecolor{darkgreen}{HTML}{006400}

%\usepackage[usenames,dvipsnames]{color} % more flexible names for syntax highlighting colors


\lstset{
basicstyle=\ttfamily, 
columns=fullflexible, % make sure to use fixed-width font, CM typewriter is NOT fixed width
numbers=left, 
numberstyle=\small\ttfamily\color{gray},
stepnumber=1,              
numbersep=10pt, 
numberfirstline=true, 
numberblanklines=true, 
tabsize=4,
lineskip=-1.5pt,
extendedchars=true,
breaklines=true,        
keywordstyle=\color{skyblued}\bfseries,
identifierstyle=, % using emph or index keywords
commentstyle=\sffamily\color{darkgreen},
stringstyle=\color{chameleond},
showstringspaces=false,
showtabs=false,
upquote=false,
texcl=true % interpet comments as LaTeX
}


\lstset{literate=
  {α}{{$\alpha$}}1 {Δ}{{$\Delta$}}1
  {á}{{\'a}}1 {é}{{\'e}}1 {í}{{\'i}}1 {ó}{{\'o}}1 {ú}{{\'u}}1
  {Á}{{\'A}}1 {É}{{\'E}}1 {Í}{{\'I}}1 {Ó}{{\'O}}1 {Ú}{{\'U}}1
  {à}{{\`a}}1 {è}{{\`e}}1 {ì}{{\`i}}1 {ò}{{\`o}}1 {ù}{{\`u}}1
  {À}{{\`A}}1 {È}{{\'E}}1 {Ì}{{\`I}}1 {Ò}{{\`O}}1 {Ù}{{\`U}}1
  {ä}{{\"a}}1 {ë}{{\"e}}1 {ï}{{\"i}}1 {ö}{{\"o}}1 {ü}{{\"u}}1
  {Ä}{{\"A}}1 {Ë}{{\"E}}1 {Ï}{{\"I}}1 {Ö}{{\"O}}1 {Ü}{{\"U}}1
  {â}{{\^a}}1 {ê}{{\^e}}1 {î}{{\^i}}1 {ô}{{\^o}}1 {û}{{\^u}}1
  {Â}{{\^A}}1 {Ê}{{\^E}}1 {Î}{{\^I}}1 {Ô}{{\^O}}1 {Û}{{\^U}}1
  {œ}{{\oe}}1 {Œ}{{\OE}}1 {æ}{{\ae}}1 {Æ}{{\AE}}1 {ß}{{\ss}}1
  {ű}{{\H{u}}}1 {Ű}{{\H{U}}}1 {ő}{{\H{o}}}1 {Ő}{{\H{O}}}1
  {ç}{{\c c}}1 {Ç}{{\c C}}1 {ø}{{\o}}1 {å}{{\r a}}1 {Å}{{\r A}}1
  {€}{{\EUR}}1 {£}{{\pounds}}1
}

% \lstset{language=bash,
% stringstyle=\ttfamily,
% basicstyle=\footnotesize, 
% showstringspaces=false,
% breaklines=true,
% }

\lstdefinestyle{numbers} {numbers=left, stepnumber=5, numberstyle=\tiny, numbersep=10pt}
% \lstdefinestyle{lc3} {language=[x86masm]Assembler,style=numbers,frame=lines,showstringspaces=false,keywords={BR,LD,BRz,BRn,BRzp,BRnz,BRnzp,LEA,HALT,RET,.END,.ORIG,.FILL,.STRINGZ,PUTS,ADD,AND,OR,OUT,NOT,LDR,STR,JSR,NOP,GETC},showstringspaces=false,keepspaces=true,flexiblecolumns=true,deletekeywords={end}}

\lstdefinestyle{target}
{language=[x86masm]Assembler,style=numbers,frame=lines,showstringspaces=false,keywords={add,
  jump, snif,call,return,.string,.reserve,.word,.set,.align16,letl,leth,.let,and,or,wmem,rmem,asr,xor,sub,sgt,sle,slt,slt},showstringspaces=false,keepspaces=true,flexiblecolumns=true,deletekeywords={end},breaklines=true}



%\definecolor{chameleond}{HTML}{4E9A06}
%\definecolor{skyblued}{HTML}{204A87}


\lstdefinestyle{digmips} {language=[x86masm]Assembler,style=numbers,frame=lines,showstringspaces=false,keywords={add,sub,ld,ldi,ble,j,jmp,st},showstringspaces=false,keepspaces=true,flexiblecolumns=true,deletekeywords={end},keywordstyle=\color{skyblued},commentstyle=\color{red}\bf,morecomment=[l][\color{red}]{//}}


\lstdefinestyle{lc3} {language=[x86masm]Assembler,style=numbers,frame=lines,showstringspaces=false,keywords={BR,LD,BRz,BRn,BRzp,BRnz,BRnzp,LEA,HALT,RET,.END,.ORIG,.FILL,.STRINGZ,PUTS,ADD,AND,OR,OUT,NOT,LDR,STR,JSR,NOP,GETC},showstringspaces=false,keepspaces=true,flexiblecolumns=true,deletekeywords={end}}


\lstdefinelanguage{[Objective]Caml}
{basicstyle=\ttfamily,
keywordstyle=\itshape\bfseries,
identifierstyle=,
stringstyle=\itshape,
commentstyle=\itshape,
flexiblecolumns=true,
escapechar=\%}

\lstdefinelanguage{julia}
{
  keywordsprefix=\@,
  morekeywords={
    exit,whos,edit,load,is,isa,isequal,typeof,tuple,ntuple,uid,hash,finalizer,convert,promote,
    subtype,typemin,typemax,realmin,realmax,sizeof,eps,promote_type,method_exists,applicable,
    invoke,dlopen,dlsym,system,error,throw,assert,new,Inf,Nan,pi,im,begin,while,for,in,return,
    break,continue,macro,quote,let,if,elseif,else,try,catch,end,bitstype,ccall,do,using,module,
    import,export,importall,baremodule,immutable,local,global,const,Bool,Int,Int8,Int16,Int32,
    Int64,Uint,Uint8,Uint16,Uint32,Uint64,Float32,Float64,Complex64,Complex128,Any,Nothing,None,
    function,type,typealias,abstract
  },
  sensitive=true,
  morecomment=[l]{\#},
%  morecomment=[s]{# =}{=#},
  morestring=[b]',
  morestring=[b]" 
}

\lstdefinelanguage{cplus}
{
        frame=lines,
	language=C++,
	basicstyle=\tt,
        morecomment=[l][\color{magenta}]{\#},
	keywordstyle=\color{chameleond},
	identifierstyle=\color{black},
        stringstyle=\color{skyblued},
        commentstyle=\color{red},
	tabsize=4,
}


\lstdefinelanguage{mypython}
{
%        frame=lines,
	language=Python,
	basicstyle=\tt,
        morecomment=[l][\color{magenta}]{\#},
	keywordstyle=\color{chameleond},
	identifierstyle=\color{black},
        stringstyle=\color{skyblued},
        commentstyle=\color{red},
	tabsize=4,
        breaklines=true,
}


\lstdefinelanguage{flexbison}
{
        frame=lines,
	language=C,
	basicstyle=\tt,
        keywords={yylex,yyparse,yyline,yytext,yyin},
        morecomment=[l][\color{magenta}]{\#,},
	keywordstyle=\color{chameleond},
	identifierstyle=\color{black},
        stringstyle=\color{skyblued},
        commentstyle=\color{red},
	tabsize=4,
}


\lstdefinelanguage{antlr}
{
%  frame=lines,
  language=Java,
  basicstyle=\tt\scriptsize,
  breaklines=true,%                                      allow line breaks
  moredelim=[s][\color{blue!50!black}\ttfamily]{'}{'},% single quotes in green
  moredelim=*[s][\color{black}\ttfamily]{options}{\}},%  options in black (until trailing })
  commentstyle={\color{gray}\itshape},%                  gray italics for comments
  morecomment=[l]{//},%                                  define // comment
  emph={%
    STRING%                                            literal strings listed here
  },emphstyle={\color{blue}\ttfamily},%              and formatted in
                                %              blue
  keywords={grammar,header,members,skip},
  alsoletter={:,|,;},%
  morekeywords={:,|,;},%                                 define the special characters
  keywordstyle={\color{chameleond}},%                         and format them in black
}


\lstdefinelanguage{mycaml}
{       language=Caml,
        upquote=true,
        columns=flexible,
        keepspaces=true,
        breakindent=0pt,
        basicstyle=\ttfamily,
        breaklines=true,
        keywordstyle=\color{red},
        commentstyle=\color{darkgreen},
        tabsize=2,
        escapechar=\%,
        escapebegin=\color{blue}
        }

\lstdefinelanguage{myhaskell}
{       language=Haskell,
        upquote=true,
        columns=flexible,
        keepspaces=true,
        breakindent=0pt,
        basicstyle=\ttfamily\scriptsize,
        breaklines=true,
        keywordstyle=\color{red},
        commentstyle=\color{darkgreen},
        tabsize=2,
        escapeinside={!}{!},
        escapebegin=!\endgraf\color{gray},
        }



\newcommand\ocamli[1]{\lstinline[language=mycaml,basicstyle=\ttfamily\normalsize]{#1}}
\newcommand\llvminline[1]{\lstinline[language=LLVM,basicstyle=\ttfamily\normalsize]{#1}}
\newcommand\llvmfile[1]{\lstinputlisting[language=LLVM]{#1}}

\lstnewenvironment{llvmc}{
  \lstset{language=llvm,mathescape,
    basicstyle=\ttfamily\small,
    aboveskip={0\baselineskip},
    belowskip=0\baselineskip
  }}{}


\lstdefinelanguage{lustre}
{
  basewidth={0.5em,0.5em},
  % list of keywords
  morekeywords={
    pre,
    fby,
    ->,
    current,
    when,
    node, 
    returns,
    let,
    tel,
    var,
    if,
    then,
    int,
    bool,
  },
  sensitive=false, % keywords are not case-sensitive
  escapeinside={!}{!},          % if you want to add LaTeX within your code
  morecomment=[l]{//}, % l is for line comment
  morecomment=[s]{/*}{*/}, % s is for start and end delimiter
  morestring=[b]", % defines that strings are enclosed in double quotes
  basicstyle=\ttfamily,
  keywordstyle=\bfseries\color{blue!80!black},
  keywordstyle=[2]\bfseries\color{red!50!white},
  commentstyle=\itshape\color{purple!40!black},
  %identifierstyle=\color{blue!80!black},
  stringstyle=\color{orange}
}


\newcommand{\arthur}[1]{\textcolor{violet}{#1}}
\newcommand{\marie}[1]{\textcolor{red}{#1}}
\newcommand{\prof}[1]{\textcolor{gray}{#1}}

% individualisation des parametres de la page
\parskip8pt
\setlength{\topmargin}{-25mm}
\setlength{\textheight}{250mm}

%
% -----------------------------------------------------------------------------------------------------------------------------------------------------
%

\newtheorem{thm}{Théorème}

%document
\title{Mini}
\begin{document}

\vspace{50mm}
{\large
\begin{center}
  Université de Nantes --- UFR Sciences et Techniques\\
  Master informatique parcours ``optimisation en recherche opérationnelle (ORO)''\\
  Année académique 2018-2019
  \vspace{30mm}
 
  { \LARGE
 
     Complexité des algorithmes\\
     \vspace{5mm}
 
     {\huge \textbf{preuve du Master Théorème}}
     \vspace{5mm}
 
     Marie \textsc{Humbert--Ropers}$^1$ --  Arthur \textsc{Gontier}$^2$
     \vspace{50mm}
  
     \today
  }  
\end{center}
}

\vfill
\break

La définition prise du master Théorème est la suivante :
\begin{thm}
  Master théorème :


  Soit n $\in$ $\mathbb{N^*}$ et, C(n) une fonction telle que
\begin{equation}
    C(n) \leq a.C(\frac{n}{b})+f(n)
\end{equation}
\begin{equation}
    $$C(1)=c$$
\end{equation}
  où $a\geq 1, b \geq 2, c \geq 0$.

  Si $f(n) = \Theta (n^d)$ avec $d \geq 0$, alors :

  \begin{itemize}
  \item si $a<b^d, C(n) \in \mathcal{O}(n^d)$
  \item si $a=b^d, C(n) \in \mathcal{O}(n^d.log(n))$
  \item si $a>b^d, C(n) \in \mathcal{O}(n^{log_b(a)})$
  \end{itemize}
  
\end{thm}
\begin{proof}
  \`A partir de l'équation $(1)$, il est possible d'établir la relation suivante, en remplaçant $C(\frac{n}{b})$ :
  $$C(n)
  \leq aC(\frac{n}{b})+f(n)
  \leq a(aC(\frac{n}{b^2})+f(\frac{n}{b}))+f(n)$$

De la même manière, on établit que :

  $$C(n)
  \leq aC(\frac{n}{b})+f(n)
  \leq a(aC(\frac{n}{b^2})+f(\frac{n}{b}))+f(n)
  \leq a(a(aC(\frac{n}{b^3})+f(\frac{n}{b^2}))+f(\frac{n}{b})))+f(n)$$

  Ainsi, récursivement, on établit pour la $k^{ieme}$ itération la relation suivante :
\begin{equation}
  C(n) \leq a^kC(\frac{n}{b^i})+\sum_{i=0}^{k-1}a^if(\frac{n}{b^i})
\end{equation}

Dans l'objectif de majorer $C(n)$, on s'intéresse à l'itération $k$ tel que $n \leq b^{k}$. Nous partons de l'hypothèse que n est une puissance de b. Nous nous intéressons donc uniquement au cas de $n = b^{k}$, dans le but de faire apparaître $C(1)$ dans l'inéquation (2):

\begin{equation}
  C(n) \leq a^kC(1)+\sum_{i=0}^{k-1}a^if(\frac{n}{b^i})
\end{equation}

Or, d'après l'énoncé, $C(1) = c$ avec c une constante. On note aussi que   $$\frac{n}{b^k}=1 \Leftrightarrow n=b^k \Leftrightarrow ln(n)=k.ln(b) \Leftrightarrow k=\frac{ln(n)}{ln(b)}$$ . 

D'où : $a^{k} = a^{\frac{ln(n)}{ln(b)}} = n^{\frac{ln(n)}{ln(b)}} = n^{log_{b}(a)}$

Ainsi, $a^kC(1)$ est en $(\Theta (n^{log_{b}(a)}))$. Il est donc nécessaire de s'intéresser au second terme de la partie de droite de (4). Nous distinguons alors 3 cas.\\

\begin{itemize}
  \item $a<b^d$ \\
	Dans ce cas, $a<b^d \Leftrightarrow ln(a) < d \times ln(b) \Leftrightarrow \frac{ln(a)}{ln(b)} < d \Leftrightarrow log_{b}(a) < d$, car $a \leq 2$ .


On sait que $f(n) \in \Theta (n^d)$ avec $d \geq 0$ donc, ici, nous nous intéressons asymptotiquement au terme suivant :
\begin{equation}
	\sum_{i=0}^{k-1}a^i(\frac{n}{b^i})^d
\end{equation}
Or, 
\begin{equation}
\sum_{i=0}^{k-1}a^i(\frac{n}{b^i})^d = \sum_{i=0}^{k-1}(\frac{a}{b^d})^in^d = n^d * \sum_{i=0}^{k-1}(\frac{a}{b^d})
\end{equation}
Comme $a<b^d$, on remarque que l'on est en présence d'une suite géométrique de raison $\frac{a}{b^d}$. Il est donc possible de l'écrire :
\begin{equation}
\sum_{i=0}^{k-1}a^i(\frac{n}{b^i})^d = n^d * \frac{1 - (\frac{a}{b^d})^k}{1 - \frac{a}{b^d}} = n^d * \frac{1 - (\frac{a}{b^d})^{log_{b}(n)}}{1 - \frac{a}{b^d}}
\end{equation}
Ainsi,
\begin{equation}
C(n) \in \mathcal{O}(n^{log_{b}(a)}c+n^d\frac{1-(\frac{a}{b^d})^{{log_{b}(n)}}}{1-\frac{a}{b^d}})
\end{equation}
Le dénominateur est une constante et, comme $k = log_{b}(n)$ et que $a<b^d$, alors asymptotiquement, $(\frac{a}{b^d})^k$ tend vers 0 et, le numérateur tend vers 1. Ainsi, on peut réduire le facteur de droite de l'inéquation précédente,  et comme $log_{b}(a) < d$, par suite, $C(n) \in \mathcal{O}(n^{d})$\\

  \item $a=b^d$ \\
	Dans ce cas, $a=b^d \Leftrightarrow ln(a) = dln(b) \Leftrightarrow \frac{ln(a)}{ln(b)} = d $. 
On sait que $f(n) = \Theta (n^d)$ avec $d \geq 0$ donc, ici, par substitution , on étudie asymptotiquement le comportement de :
	$$\sum_{i=0}^{k-1}a^i(\frac{n}{b^i})^d$$
Ainsi, on remarque que, comme $a=b^d$ :
$$\sum_{i=0}^{k-1}a^i(\frac{n}{b^i})^d = \sum_{i=0}^{k-1}(\frac{a}{b^d})^in^d = n^d\sum_{i=0}^{k-1}1 = n^d*k = n^d*log_{b}(n) $$
Or, $k = log_{b}(n)$ et $ d = log_{b}(a) $, nous obtenons donc l'expression suivante :

Ainsi,
\begin{equation}
C(n) \in \mathcal{O}(n^{log_{b}(a)}c+n^{log_{b}(a)}*log_{b}(n))
\end{equation} 
Comme $log_{b}(a) = d$, $(n^{d}*ln(n))$ majore asymptotiquement $n^{log_{b}(a)}c$, d'où :  $C(n) \in \mathcal{O}(n^{d}*ln(n))$\\

  \item $a>b^d$ \\
On sait que $f(n) = \Theta (n^d)$ avec $d \geq 0$, et $f(\frac{n}{b^i}) = \mathcal{O}((\frac{n}{b^i})^d)$ donc, asymptotiquement :
\begin{equation}
	\sum_{i=0}^{k-1}a^i(\frac{n}{b^i})^d 
\end{equation}

Par suite, 
\begin{equation}
	\sum_{i=0}^{k-1}a^i(\frac{n}{b^i})^d = a^k*\sum_{i=0}^{k-1}\frac{a^in^d}{b^{d*i}a^k} = a^k*\sum_{i=0}^{k-1}\frac{n^d}{b^{d*i}a^{k-i}}
\end{equation}

En substituant les i par j, tel que $j = k-i$ et en sommant en "sens inverse" :
\begin{equation}
	\sum_{i=0}^{k-1}a^i(\frac{n}{b^i})^d = a^k*\sum_{j=0}^{k-1}\frac{n^d}{b^{d*(k-j)}a^{j}} = a^k*\sum_{j=0}^{k-1}\frac{n^db^j}{b^{d*k}a^{j}} = a^k*\sum_{j=0}^{k-1}\frac{n^db^{j*d}}{b^{d*k}a^{j}}
\end{equation}

  Or :
  $$ b^k=(e^{ln(b)})^{\frac{ln(n)}{ln(b)}}=e^{\frac{ln(b)ln(n)}{ln(b)}}=n$$

Cela nous permet de simplifier l'expression suivante :

\begin{equation}
	\sum_{i=0}^{k-1}a^i(\frac{n}{b^i})^d = a^k*\sum_{j=0}^{k-1}\frac{b^{j*d}}{a^j}
\end{equation}

Par hypothèse,  $a>b^d$, donc $ \frac{b^d}{a} \leq 1$ . En utilisant une suite géométrique de raison  $ \frac{b^d}{a}$ : 

\begin{equation}
	\sum_{i=0}^{k-1}a^i(\frac{n}{b^i})^d = a^k*(\frac{1-(\frac{b^d}{a})^k}{1-\frac{b^d}{a}})
\end{equation}

Comme $k = log_{b}(n)$ et que $a>b^d$, alors asymptotiquement, $(\frac{b^d}{a})^k$ tend vers 0 et, le fraction tend vers une constante. 
Donc, ce terme aussi est d'ordre $\mathcal{O}(n^{log_b(a)})$ tout comme $a^kc$

Donc, $$C(n) \in \mathcal{O}(n^{log_b(a)})$$


\end{itemize}


Il est donc nécessaire dans un second temps de généraliser cette preuve à tout nombre n qui n'est pas une puissance de b. Pour ce faire, il faut se préoccuper de trouver des bornes à $aC(\lfloor \frac{n}{b} \rfloor)+f(n)$ et à $aC(\lceil \frac{n}{b} \rceil)+f(n)$. La démonstration est ensuite sensiblement la même.

\end{proof}

\end{document}


